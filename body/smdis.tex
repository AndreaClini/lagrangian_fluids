\section{Standard Model Deep Inelastic Scattering: a review}
\label{sec:smdis}
In this section we review under the hat of the OPE approach the main properties of SM deep inelastic scattering. DIS in SM processes is more than a more simple avatar of the dark signal that will be central in this work, but it will contribute as a source of background as we discuss in \cref{sec:pheno}. In particular, we focus on DIS processes where a neutrino hits a proton or a neutron, which naively mimic the signal of a DS particles hitting a nucleon of the detector and inducing its fragmentation. 

\subsection{Standard Model Neutral Current Inelastic Scattering}
For simplicity, we focus on neutrino scattering over a proton--results for neutrons are obtained via isospin symmetry. --of the target, exchanging an off-shell $Z$ boson, a good approximation up to \begin{equation}
    m_Z\sim 90{\rm\, GeV}\to s\sim 8100{\rm\, GeV}^2\to E_\nu\lesssim \frac{s}{2m_p}\sim 4{\rm\, TeV}
\end{equation}
as it is for the next generation neutrino beams. For neutrino energy $E_\nu\gtrsim 10{\rm\,GeV}$~\cite{Formaggio:2012cpf}, the interactions resolves the partonic structure of the proton and the $Z$ boson is exchanged between neutrino and a quark, which is kicked away from the proton, as shown in \cref{fig:smdis}. More in detail, we define $Q^2=-q^2=-(k_1-k_2)^2\geq 0$ the virtuality of the $Z$ boson. For $Q^2\gg m_p^2$, the quark is kicked away from the proton leading to an inelastic scattering. This produces gluons and quark-antiquark pairs to create final states with neutral color charge. In the following description of DIS, we will neglect resonance production; hence our computation will be reliable only for hadronic invariant masses $m_X^2=(\sum_{f} p_f)^2\gg m_p^2$, with $f$ being all the mesons and barions produced at the and of the showering process.
\begin{figure}[t!]
    \centering
    \includegraphics[width=0.49\linewidth]{images/smdisir.pdf}
    \includegraphics[width=0.49\linewidth]{images/smdisquark.pdf}
    \caption{Picture of a SM DIS event mediated by an off-shell $Z$ boson: {\bf left} from the infrared proton or neutron perspective; {\bf right} from the ultraviolet quark perspective. Notice that $q$ is space-like. The hadronic final state will be populated by one or more jets in the direction of the exchanged momentum.}
    \label{fig:smdis}
\end{figure}
Since the process involves a constituent of the proton, in order to describe it one needs to understand the structure of the nucleon. We analyze the problem in the proton-neutrino center of mass reference frame. At energies well above the proton's mass, we can assume $p_3^2 = 0$, with $p_3$ being the four-momentum of the proton. Additionally, the constituents of the proton will have collinear light-like momentum because the exchange of relevant transverse momentum, which proceeds via hard gluon exchange, is suppressed by the running of the strong coupling at high energies. Thus, the interaction involves a constituent that carries a fraction $\xi$ of the proton momentum:
\begin{equation}
    k_3=\xi p_3 ,\quad \xi \in [0,1].
    \label{eq:defx}
\end{equation}
Neglecting gluon exchange during the interaction, the DIS cross-section can be expressed as the cross-section for the neutrino interaction with a quark carrying an energy fraction \(\xi\), multiplied by the probability of finding that quark with that specific energy inside the proton, integrated over \(\xi\)~\cite{Peskin:1995ev}. The constituents of the proton—quarks, antiquarks, and gluons—are collectively referred to as partons. The probabilities of finding a parton of species \(i\) carrying an energy fraction \(\xi\) are described by the Parton Distribution Functions (PDFs), $f_i(x)$. Given that the proton is a strongly coupled bound state, the PDFs cannot be calculated using perturbative Quantum Chromodynamics (QCD). However, some basic requirements on the PDFs can still be extracted from simple properties of the process. i) The proton is a bound state with valence quarks $uud$, it contains an excess of 2 $u$-quarks and 1 $d$-quark and as a consequence the normalization of the PDFs should be
    \begin{align}
        \int_0^1 & d\xi[f_u(\xi)-f_{\bar u}(\xi)]=2\,\label{eq:normcond1} ,\\
        \int_0^1 & d\xi[f_d(\xi)-f_{\bar d}(\xi)]=1.
        \label{eq:normcond2}
    \end{align}
ii) Since the sum of the momenta carried by the partons should be equal to the momentum of the proton, we get
\begin{equation}
    \int_0^1d\xi \xi\sum_{\text{partons},i}f_i(\xi)=1.
    \label{eq:normcond3}
\end{equation}
We now explore the connection between DIS events and proton structure. Since neutrinos have zero electric charge, their interactions proceed via weak forces. We consider the leptonic final state to be a neutrino since this scenario will be the background of the DS process considered in this work.

\subsubsection{From partonic to nucleon cross-section: the UV approach}

The interaction lagrangian for the neutrino neutral currents with the quarks for exchanged momentum $Q^2\ll M^2_Z$ is \cite{Erler:2013xha,Conrad:1997ne}
\begin{equation}
    \mathcal{L}_{\text{int}}=\frac{2}{ v^2} \bar\nu_L \gamma_\mu \nu_L \sum_f\left( \ell_f \bar q_{f,L} \gamma^\mu q_{f,L} + r_f \bar q_{f,R} \gamma^\mu q_{f,R}\right),\label{eq:neutralcurrent}
\end{equation}
where the overall cut-off suppression is controlled by the EW VEV $v$ which can be rewritten in terms of the Fermi constant $G_F\equiv1/(\sqrt{2} v^2)$. We also defined the \textit{charges} with respect to the $Z$ current as 
$\ell_f=T_3^f-Q_f\sin^2\theta_W$ and $r_f= Q_f \sin^2\theta_W$, with $T_3^f$ being the isospin eigenvalue of the fermion $f$, $Q_f$ its electric charge, and $\sin^2\theta_W=0.23$ is the squared sine of the Weinberg angle. 

In the center of mass (CM) of the quark-neutrino interaction and assuming that all the particles are massless, it is easy to work out the formula for the cross-section\footnote{Note that since only left-handed neutrinos exist, there is no average on the initial spin states of the neutrinos. We still have to average over the quark initial spins. }. We define the squared matrix element in Eq.~\eqref{eq:neutralcurrent} involving respectively left-handed and right-handed quarks, and fixing the $Z$ charges $r_f=\ell_f=1$ for simplicity, as
\begin{align}
    \mathcal M_L&=\bar \nu(k_2)\gamma^\mu(1-\gamma_5)\nu(k_1)\bar q(p_4)\gamma_\mu(1-\gamma_5)q(p_3),\\
    \mathcal M_R&=\bar \nu(k_2)\gamma^\mu(1-\gamma_5)\nu(k_1)\bar q(p_3)\gamma_\mu(1+\gamma_5)q(p_4).
\end{align}
Averaging over the initial spins and summing over the final ones, the fully differential cross-section is
\begin{align}
    d\sigma(\nu q_f\rightarrow \nu q_f)&=\frac{|\mathcal{\bar M}_L|^2+|\mathcal{\bar M}_R|^2}{2\hat s}\frac{d^3k_2}{2\epsilon_2(2\pi)^3}\frac{d^3k_4}{2\epsilon_4 (2\pi)^3}(2\pi)^4\delta^{(4)}(k_1+k_3 -k_2-k_4)\, \notag\\&\equiv d\sigma_L(\nu q_f\rightarrow \nu q_f)+d\sigma_R(\nu q_f\rightarrow \nu q_f),
    \label{eq:totaldisxs}
\end{align}
where $\epsilon_i$ are the energies of the outgoing particles, $\hat s$ is the energy of the process in the quark-neutrino CM reference frame, while $k_3,~ k_4$ are the initial and final momentum of the quark that absorbs the boson. According to the quark helicity selected by the projectors $(1\pm\gamma_5)$, we separate $d\sigma$ of Eq.~\eqref{eq:totaldisxs} in two contributions. Integrating over one of the outgoing momenta to solve the Dirac delta and using that $\hat s=4 \epsilon_2 \epsilon_4$, we get $\epsilon_{2,4}=\sqrt{\hat s}/2$ so that
\begin{align}
\frac{d\sigma_L}{d\hat s}&=\frac{G_F^2}{\pi}\\
     \frac{d\sigma_R}{dk_4d\cos\theta}&=\frac{G_F^2}{(2\pi)\hat s^2}\hat u^2\hat s \delta(\sqrt{\hat s}/2 -k_4)\,.
     \label{eq:rightdiffxsdis}
\end{align}
where $\hat u=-2 k_3\cdot k_2$ is the Mandelstam variable.

It is useful to express the two cross-sections in terms of dimensionless combinations of kinematics variables. Together with $\xi$, which is defined in Eq.~\eqref{eq:defx}, we define
\begin{equation}
    y\equiv\frac{2p_3\cdot q}{s}=\frac{k_3\cdot(k_1-k_2)}{k_3\cdot k_1}=\frac{\hat s+\hat u}{\hat s}\implies \frac{\hat u}{\hat s}=-(1-y).
    \label{eq:defy}
\end{equation}
\(\xi\equiv x\) and \(y\) are the crucial kinematics variables in describing the scattering interaction. The variable \(x\), known as Bjorken \(x\) and defined in Eq.~\eqref{eq:defx}, represents the fraction of the proton's momentum carried by the struck parton. A first consequence of Eq.~\eqref{eq:defy} is that 
\begin{equation}
   x=\frac{Q^2}{2p_3\cdot q}.
\end{equation}
In the proton's rest frame, where \( p_3 = (m_p, \textbf{0}) \), the scalar product \( p_3 \cdot q \) becomes
\begin{equation}
p_3 \cdot q = m_p \nu,
\end{equation}
where \(\nu\) is the energy transfer from the lepton to the proton. Thus, the expression for \(x\) becomes:
\begin{equation}
x = \frac{Q^2}{2 m_p \nu}.
\label{eq:xincm}
\end{equation}
When \(x \to 0\), we have either \(Q^2 \to 0\) (small momentum transfer) or \(\nu \to \infty\) (high-energy scattering) and represents scattering off a parton that carries an extremely small fraction of the proton's momentum.
The upper bound is given by \(x = 1\), when the parton carries all of the proton’s momentum. This occurs when the struck parton is a valence quark and the entire proton's momentum is transferred to it.

The inelasticity variable \(y\) can also be written as \(y = \frac{\nu}{E}\), where \(\nu = E - E'\) is the energy transferred to the hadronic system, and \(E\) is the initial lepton energy. 

The variable \(y\) represents the fraction of the lepton’s energy transferred to the proton. When \(y \approx 0\), the scattering is nearly elastic, with minimal energy transfer, while \(y \approx 1\) corresponds to deep inelastic scattering with a significant energy transfer, leading to a detailed probe of the proton’s internal structure.

To get $d\sigma$ in terms of $(x,y)$, the first step is expressing the two cross-sections $d\sigma_L,d\sigma_R$ in terms of the Mandelstam variable $\hat t=-\hat s(1-\cos\theta_{\text{CM}})/2$. Considering the left-handed contribution, we get
\begin{equation}
    d\sigma_L=\frac{G_F^2}{\pi}\frac{\hat s}{2}d\cos\theta =\frac{G_F^2}{\pi}d\hat t.
    \label{eq:leftcurrent}
\end{equation}
Then, looking back at the definition of $\xi\equiv x,~y$ in Eq. (\ref{eq:defx}) and Eq. (\ref{eq:defy}), we obtain the following equivalences:
\begin{equation}
    Q^2=xys\implies dxd\hat t=-\frac{dQ^2}{dy}dxdy= xs dxdy.
    \label{eq:q2def}
\end{equation}
In conclusion, we have shown that the differential cross-sections for deep inelastic interaction with protons can be expressed as
\begin{align}
\frac{d^2\sigma(\nu p\rightarrow \nu X)}{dxdy}= \frac{G_F^2}{\pi} s\left(\sum_f xf_f(x)+x f_{\bar f}(x)\right)\left[(1 +(1 - y)^2\right],
\label{eq:intronuxs}
\end{align}
where the index $f$ runs over valence and \textit{sea} quarks. The differential cross-section $d\sigma/dy$ will be a parabolic function of $y$. 

Restoring the $Z$ charges $\ell_f,\ r_f$ leads to 
\begin{align}
\frac{d\sigma(\nu_\mu p \to \nu_\mu X)}{dydx} &= \frac{G_F^2 s}{\pi}  \sum_{f=u,d,s,c} \left[ \ell_f^2 + r_f^2 (1-y)^2 \right] x f_f(x) \notag\\&+ \left[ \ell_f^2 (1-y)^2 + r_f^2 \right] x f_{\bar f}(x),
\label{eq:neutralcurrentdis}
\end{align}
Eq.~\eqref{eq:neutralcurrentdis} relies on our knowledge of QCD, the theory describing the proton at high energies, since it makes explicit reference to the PDFs $f_i(x)$ that encode how the partons share the momentum inside the bound state.
\MB{is nda correct? comment with the integrals....}



\subsubsection{Deep inelastic scattering from the IR: form factors}

In the following, we will compute again the deep inelastic scattering cross section in Eq.~\eqref{eq:totaldisxs} but parametrizing our ignorance about hadronic processes in the two-point current-current correlator. We will first decompose this correlator in terms of form factors that can be easily related to the PDFs defined above. We will then use the Operator Product expansion (OPE) to derive important properties of DIS purely from a lower energy description that ignores how partons share the momentum inside the proton.

The starting point is that hadronic neutral currents $J^\mu_{f}$ do not violate the quark flavor number $f$. As a consequence, the left-handed contribution to $J^\mu_{f}$ only couples quarks of the same flavor and has the following expression:
\begin{equation}
    J^{\mu}_{fL}=\bar q_f \gamma^{\mu}P_Lq_f, \quad P_L=\frac{1-\gamma_5}{2}.
\end{equation}
We can then in general write the matrix element as 
\begin{equation}
    \mathcal{M}(\nu p\rightarrow \nu X)=\frac{G_F}{\sqrt 2}\bar \nu(p_2)\gamma_\mu(1-\gamma_5)\nu(p_1)\langle X|J^\mu_{fL}(q)|p_3\rangle,
\end{equation}
where $|p_3\rangle$ represents an initial state proton with momentum $p_3$, $X$ is a generic hadronic final state, and $q$ is the momentum exchanged in the process.

The squared matrix element, inclusive over $X$, is then
\begin{align}
     |\mathcal{M}(\nu p\rightarrow \nu X)|^2=&\text{Tr}\left(\slashed p_2\gamma_\mu(1-\gamma_5)\slashed p_1\gamma_\nu(1-\gamma_5)\right)\notag\\\times&\sum_X\int d\Pi_X\langle p_3|J^\mu_{fL}(-q)|X\rangle\langle X|J^\nu_{fL}(q)|p_3\rangle.
\end{align}
Applying the optical theorem \cite{Peskin:1995ev} to relate the inclusive cross-section to twice the imaginary part of the matrix element\footnote{The optical theorem states that if $a$ and $b$ are particle states
\begin{equation}
    \sum_X\int d\Pi_X \mathcal M^*(b\to X)M(a\to X)=-i\left[M(a\to b)-M^*(b\to a)\right] ~,
\end{equation}
where the sum over $X$ runs over all the possible final states $X$.
We are interested in the case of $a=b$ so that the r.h.s. becomes an imaginary part, while the product on the r.h.s. becomes a squared modulus.}, we get

\begin{equation}
     |\mathcal{M}(\nu p\rightarrow \nu X)|^2=\frac{G_F^2}{4}2\text{Im}\left[\text{Tr}(\slashed p_2 \gamma_\mu(1-\gamma_5)\slashed p_1\gamma_\nu(1-\gamma_5))W^{\mu\nu}_{fL}\right],
     \label{eq:matrixelementoptthm}
\end{equation}
where we defined the following dimensionless tensor which is controlled by the expectation value of the current-current two-point function on the proton scattering state:
\begin{equation}
    W^{\mu\nu}_{fL}(q, p_3)=i\int d^4 xe^{iq\cdot x}\langle p_3|T\{J^\mu_{fL}(x)J^\nu_{fL}(0)\}|p_3\rangle\, .
    \label{eq:defw}
\end{equation} 
The tensor can only be a function of the proton momentum $p_3$ and the exchanged momentum $q$ because of momentum conservation. 
The trace of the neutrino line in Eq. (\ref{eq:matrixelementoptthm}) yields instead
\begin{equation}
    \frac{G_F^2}{4}\text{Tr}\left(\slashed {p_2}\gamma_\mu(1-\gamma_5)\slashed p_1\gamma_\nu(1-\gamma_5)\right)=2(p_1^\mu p_2^\nu+p_1^\nu p_2^\mu-g^{\mu\nu}p_1\cdot p_2+i\epsilon^{\mu\nu\rho\delta}p_{1\rho} p_{2\sigma}).
    \label{eq:lepttrace}
\end{equation}
It is useful to look back at the dimensionless scattering variables $x,~y$ of Eq (\ref{eq:q2def}) and evaluate them in the rest frame of the proton, the same of the laboratory (LAB), i.e. where $p_3=(m,0,0,0)$. In such a frame we get
\begin{equation}
    x=  \frac{Q^2}{2p_3\cdot q}=\frac{2p_1 p_2(1-\cos\theta)}{2m(p_1-p_2)},\quad y=\frac{2p_3\cdot q}{2p_3\cdot p_1}=\frac{p_1-p_2}{p_1},
\end{equation}
where $\theta$ is the angle of the outgoing neutrino in the LAB frame. 


The change of variables $(p_2,\cos\theta)\rightarrow(x,y)$ has the following jacobian:
\begin{equation}
    J=\frac{2p_2}{2m(p_1-p_2)}=\frac{2p_2}{ys}\implies \frac{d^3p_2}{(2\pi)^32\epsilon_2}=\frac{2\pi p_2 dp_2d\cos\theta}{(2\pi)^32}=dxdy\frac{ys}{(4\pi)^2}.
\end{equation}
The cross-section can then be written as\footnote{The phase space of the single-particle final state representing $X$ is $d^4p_X/(2\pi)^4$ since we are integrating over all the mass shells for these hadronic states.}
\begin{equation}
\frac{d^2\sigma(\nu p\rightarrow\nu X)}{dxdy}=\frac{G_F^2y}{2\pi^2}\text{Im}\left[(p_1^\mu p_2^\nu+p_1^\nu p_2^\mu-g^{\mu\nu}p_1\cdot p_2+i\epsilon^{\mu\nu\rho\delta}p_{1\rho} p_{2\sigma})W^{\mu\nu}_{fL}(p_3,q)\right],    
\label{eq:disxswithope}
\end{equation}
If the current involved in the hadronic process is conserved, the Ward identity 
\begin{equation}
    q_\nu W^{\mu\nu}_{fL}=q_\mu W^{\mu\nu}_{fL}=0\label{eq:wardid}
\end{equation}
fixes the structure of $W^{\mu\nu}_{fL}$ in terms of scalar form factors. 

For weak currents, the expansion in terms of form factors starts with the observation that any term in $W^{\mu\nu}_{fL}$ proportional to the exchanged momentum $q^\mu$ vanishes when contracted with the lepton tensor trace of Eq. (\ref{eq:lepttrace}), since using $p_1^2=p_2^2=0$ we get
\begin{align}
    (p_{1\mu}-p_{2\mu})&(p_1^\mu p_2^\nu+p_1^\nu p_2^\mu-g^{\mu\nu}p_1\cdot p_2+i\epsilon^{\mu\nu\rho\delta}p_{1\rho} p_{2\sigma})\notag\\&=p_1\cdot p_2 p_2^\nu-p_1^\nu p_1\cdot p_2-p_1\cdot p_2 p_2^\nu+p_1\cdot p_2 p_2^\nu=0.
\end{align}
As a consequence, $W^{\mu\nu}_{fL}(p_3,q)$ has an expression in terms of three scalar form factors:
\begin{equation}
    W^{\mu\nu}_{fL}(p_3,q)=-g^{\mu\nu}W_{1fL}(p_3,q)+p_3^\mu p_3^\nu W_{2fL}(p_3,q)+i\epsilon^{\mu\nu\rho\sigma}p_{3\rho} q_\sigma W_{3fL}(p_3,q)\,
    \label{eq:introexp}
\end{equation}
\noindent where $W_{1,2}$ are parity even and related by the Ward identies in Eq.~\eqref{eq:wardid} while $W_3$ is parity odd. Contracting Eq. (\ref{eq:introexp}) with $q^\mu q^\nu$, we can derive the following relation between $W_{1,2}$:
\begin{equation}
    0=Q^2W_{1fL}(p_3,q)+(p_3\cdot q)^2W_{2fL}(p_3,q)\implies W_{2fL}(p_3,q)=-\frac{4x}{ys}W_{1fL}(p_3,q).
    \label{eq:wardid2}
\end{equation}

The form factor decomposition of the QED current two-point function is obtained by fixing $W_3=0$, since the electromagnetic current,  $\bar \psi \gamma^\mu\psi$ transforms as a four-vector under parity. Since the weak left-handed current is a combination of vector and axial current it does not transform in a simple way under parity, resulting in a parity violation contribution encoded in the $W_3$ form factor.


First, it is interesting to exploit our knowledge of QCD to evaluate the form factors in the parton model using the PDFs. This amounts to compute the diagrams shown in Fig. \ref{fig:intropartonfey}. Defining $p=\xi p_3$ the momentum carried by the quark $f$, we can write $W^{\mu\nu}=\sum_f W^{\mu\nu}_{fL}$ as
\begin{equation}
    W^{\mu\nu}\approx i\int d^4 xe^{iq\cdot x}\int_0^1d\xi\sum_ff_f(\xi)\frac{1}{\xi}\langle q_f(p)|T\left\{J^\mu_L(x)J^\nu_L(0)\right\}|q_f(p)\rangle\big{|}_{p=\xi p_3},
\end{equation}
where the factor $1/\xi$ gives the normalization of the initial state phase space in terms of partons: 
\begin{equation}
    \frac{1}{2s}\rightarrow \frac{1}{2s\xi}.
\end{equation}
\begin{figure}
    \centering
    \includegraphics[width=\textwidth]{intro/opefeynman.pdf}
    \caption{Feynman diagram contributing to the amplitude. Note that the second one is obtained from the first by exchanging $q,\mu$ with $-q,\nu$. The Fourier transform of the expectation value of two currents can be interpreted as the amplitude of a process involving a quark with initial momentum $p$, that interacts with two bosons and escapes with momentum $p$. At the leading order in QCD, this corresponds to an absorption of one such boson, leading to a virtual propagation of the quark that then emits again a boson.}
    \label{fig:intropartonfey}
\end{figure}
The first diagram on the r.h.s. of Fig. \ref{fig:intropartonfey} is
\begin{align}
    2i\int_0^1 d\xi[&f_f(\xi)\frac{1}{\xi}\bar u_f(p)\gamma^\mu\frac{1-\gamma_5}{2}\frac{i}{\slashed p+\slashed q+i\epsilon}\gamma^\nu\frac{1-\gamma_5}{2}u_f(p)+\notag\\&
    f_{\bar f}(\xi)\frac{1}{\xi}\bar v_{\bar f}(p)\gamma^\nu\frac{1-\gamma_5}{2}\frac{i}{\slashed p+\slashed q+i\epsilon}\gamma^\mu\frac{1-\gamma^5}{2}v_{\bar f}(p) ].
    \label{eq:ffloqcd}
\end{align}
Note the different order between the indices of the two lines due to the different ways in which we follow the current flow for particles in the first line and antiparticles in the second one.
Summing over the final states and averaging over the initial ones we obtain
\begin{align}
    \int_0^1 d\xi [&f_f(\xi)\frac{1}{\xi}\text{Tr}(\slashed p\gamma^\mu\frac{1-\gamma_5}{2}(\slashed p+\slashed q)\gamma^\nu\frac{1-\gamma_5}{2})\notag\\&+f_{\bar f}(\xi)\frac{1}{\xi}\text{Tr}(\slashed p\gamma^\nu\frac{1-\gamma_5}{2}(\slashed p+\slashed q)\gamma^\mu\frac{1-\gamma_5}{2})]\frac{-1}{2p\cdot q+q^2+i\epsilon}.
\end{align}
Neglecting $q^\mu$ terms and using $p^2=0$, the non-vanishing terms in the traces are
\begin{equation}
-2i\epsilon^{\mu\nu\alpha\beta}p_{\alpha}q_{\beta}+4p^{\mu}p^{\nu}-2g^{\mu\nu}p\cdot q
\end{equation}
for quark, while for antiquark we get the same result with $\mu,\nu$ exchanged.
All in all for the first diagram we find
\begin{align}
    \int_0^1 \frac{d\xi}{\xi}[(f_f(\xi)&+f_{\bar f}(\xi))(4\xi^2p_3^\mu p_3^\nu -2g^{\mu\nu}\xi p_3\cdot q)\notag\\& +(f_f\xi)-f_{\bar f}(\xi))2i{\epsilon^{\mu\nu}}_{\rho\sigma}p_3^{\rho}q^\sigma]\frac{-1}{2\xi p_3\cdot q+q^2+i\epsilon},
\end{align}
where the $\epsilon$ term gets a minus sign in front of the parton distribution function of the antiquark because of the different order of the $\mu\nu$ indices. 


The second diagram of the r.h.s. of Fig. \ref{fig:intropartonfey} have a denominator of the form ${2p\cdot q-q^2+i\epsilon}$, with $q^2<0$, meaning that we can safely take the limit $\epsilon=0$ and leave the diagram with no imaginary part. As a consequence, it does not contribute to the imaginary part of $W^{\mu\nu}$ and we can neglect it.

Comparing this expression with the expansion of $W^{\mu\nu}$ in terms of form factors in Eq. (\ref{eq:introexp}), we get 
\begin{align}
    \text{Im}W_{1fL}\approx 2p_3\cdot q\int_0^1 d\xi [f_f(\xi)+f_{\bar f}(\xi)]\text{Im}\left(\frac{-1}{2\xi p_3\cdot q+q^2+i\epsilon}\right)
    \label{eq:w1im}
\end{align}
from the term proportional to the metric, while from the term proportional to $p_3^\mu p_3^\nu$
\begin{align}
    \text{Im}W_{2fL}\approx\int_0^1 d\xi 4\xi[f_f(\xi)+f_{\bar f}(\xi)]\text{Im}\left(\frac{-1}{2\xi p_3\cdot q+q^2+i\epsilon}\right),
    \label{eq:w2im}
\end{align}
and from the term proportional to the Levi-Civita tensor
\begin{align}
    \text{Im}W_{3fL}\approx2\int_0^1 d\xi [f_f(\xi)-f_{\bar f}(\xi)]\text{Im}\left(\frac{-1}{2\xi p_3\cdot q+q^2+i\epsilon}\right).
    \label{eq:w3im}
\end{align}
The imaginary part inside the integrals in Eqs. (\ref{eq:w1im})-(\ref{eq:w3im}) can be computed as
\begin{equation}
    \text{Im}\left(\frac{-1}{2\xi p_3\cdot q+q^2+i\epsilon}\right)=\pi\delta(2\xi p_3\cdot q-Q^2)=\frac{\pi}{ys}\delta(\xi-x),
\end{equation}
where we used $ys=2p_3\cdot q$.
We can conclude that at leading order in QCD:
\myboxformula{\begin{align}
    \text{Im}W_{1fL}\approx& \pi[f_f(x)+f_{\bar f}(x)]\equiv \pi F_f^+(x) \label{eq:introqcd1},\\
    \text{Im}W_{2fL}\approx&\frac{4\pi x}{ys}[f_f(x)+f_{\bar f}(x)]=\frac{4 x}{ys}\text{Im}W_{1fL},\label{eq:introqcd2}\\
    \text{Im}W_{3fL}\approx& \frac{2\pi}{ys}[f_f(x)-f_{\bar f}(x)]\equiv \frac{2\pi}{ys} F_f^-(x).
    \label{eq:introqcd3}
\end{align}}
Eqs. (\ref{eq:introqcd1})-(\ref{eq:introqcd3}) link the scalar form factors of Eq. (\ref{eq:introexp}) to quarks PDFs. Note that Eq. (\ref{eq:introqcd2}) is a consequence of the Ward Identity, and in particular of Eq. (\ref{eq:wardid2}). In fact, in the $x$ complex plane the different power to which $x$ is elevated is reflected in a $-$ sign at the level of the imaginary part.

