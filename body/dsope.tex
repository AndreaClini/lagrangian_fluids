\section{Generic properties of OPE and conformal symmetry}
\label{sec:dsope}

In this section we derive the OPE for the portals considered in this work for the process of \cref{fig:dsdis}. Before deepening in the phenomenological consequences of the OPE, we review its main properties. 

\begin{figure}[t!]
    \centering
    \includegraphics[width=0.5\linewidth]{images/darkdarkdis.pdf}
    \caption{Schematic picture of a double DIS (2DIS) process--inelastic both in the SM and DS line.}
    \label{fig:dsdis}
\end{figure}

\subsection{Operator Product Expansion}\label{appendix:ope}

In this Appendix, we demonstrate the unitarity bound \cite{Mack:1975je,Minwalla:1997ka} to give a lower bound to the twist appearing in the OPE. We start with some definitions and examples of OPE and then we deal with unitarity constraints in  \cref{sec:unitarity}.

\paragraph{Definition of OPE in QFTs} A Quantum Field Theory possesses \cite{Klehfoth:2022vgx} an Operator Product Expansion for all the observables if, for any physical state $\Psi$, the expectation value of any product of local quantum field
observables $\Phi_i$ can be approximated near event $z$ as
\begin{equation}
    \langle \Psi|\Phi_1(x_1)\Phi_2(x_2)\dots \Phi_n(x_n)|\Psi\rangle \approx \sum_B C^B_{1,2\dots n}(x_1\dots x_n;z)\langle\Psi|\Phi_B(z)|\Psi\rangle.
    \label{eq:defope}
\end{equation}
$B$ labels all the observables, while the coefficients $C_{1,2\dots n}^B$ are $\mathbf{C}$-valued distributions. The symbol $\approx$ indicates the asymptotic nature of expansion, which holds as $x_i\rightarrow z$, not as the number of the terms in the series diverges.

For instance, consider Minkowski spacetime in 4 dimensions and a free massless Klein-Gordon Field $\Phi(x)$. For $x_1,x_2\rightarrow z$, the following expansion holds \cite{Hollands:2023txn,Polchinski:1994mb}:
\begin{equation}
\langle\Psi|\Phi(x_1)\Phi(x_2)|\Psi\rangle\sim\langle\Psi|\frac{1}{2\pi^2}\frac{1}{{(x_1 - x_2)^2 + i \varepsilon (x_{1}^0 - x_{2}^0)}} \textbf{1} + O(z)+\dots|\Psi\rangle,
\end{equation}
where $\textbf 1$ is the identity operator, $O\equiv \phi^2$ is a new local field and the dots represent other local operators whose coefficients vanish as $x_i\rightarrow z$. 

The coefficient of the identity operator is divergent in the working limit. The renormalization of the previous expectation value follows by defining $O=\Phi^2$, removing away the infinite fluctuations of the field. 

\paragraph{OPE in CFTs} The basic idea in the case of CFT with dimension $d\geq 2$ is that we can group local operators into multiplets \cite{Hollands:2023txn}, each multiplet being composed of a primary local operator $O$\footnote{Primary operators cannot be written as $P^\mu=i\partial^\mu$ acting on other operators.}, together with its descendants $\partial_{a_1}\dots\partial_{a_k}O$. We focus on primary operators because results for descendants can be derived by the action of $P_\mu$. The expansion of two primary operators can thus be organized as 
\begin{equation}
    O_i(x_1) O_j(x_2) = \sum_{k} \frac{f_{ijk} }{|x_1 - x_2|^{d_i + d_j - d_k}}{P_{ij}^k(x_1 - x_2; y, \partial y)} O_k(y),
\end{equation}
where $f_{ijk}$ are structure constants and $P_{ij}^k$ are differential operators that depend on the spin of the fields $l_i$, while $d_i$ are the non-negative dimensions of the fields. This OPE is fixed by the set of numbers $\{f_{ijk},l_i,d_i\}$, called OPE coefficients. In the following, we will assume that we will deal only with primary operators. 

\subsection{Unitarity via physical reasons}\label{sec:unitarity}

Unitarity bounds follow from the algebraic properties of irreducible representations of the conformal group induced on Minkowski space \cite{Mack:1975je,Minwalla:1997ka}. The same result may be obtained via physical reasons. In the following, we generalize the result of Ref. \cite{Grinstein:2008qk} to a spacetime with arbitrary dimension, requiring positivity of total cross-sections, that through optical theorem can be related to the positivity of the imaginary part of the forward amplitude \cite{Peskin:1995ev}.

\subsubsection{Scalar operator} Consider a scalar CFT operator $O(x)$ coupled to a source $J$, that may annihilates or creates particles:
\begin{equation}
    \mathcal{L}\supset g J \mathcal{O},
\end{equation}
with a scaling dimension of $\mathcal{O}$ equal to $\Delta$, and let $d$ be the dimension of the spacetime. The two-point function of $\mathcal{O}$ is fixed by the scaling symmetry: exploiting rotational and translational symmetry we can state that it is a function of $|x|$:
\begin{equation}
    \langle\mathcal{O}(x)\mathcal{O}(0)\rangle=f(|x|).
\end{equation}
Now we use the transformation properties under scaling to state that
\begin{equation}
    f(|x|)=\lambda^{2\Delta}f(\lambda|x|),
\end{equation}
and taking $\Lambda=1/|x|$ we obtain
\begin{equation}
    \langle\mathcal{O}(x)\mathcal{O}(0)\rangle=\frac{\alpha}{(x^2)^{\Delta}},
    \label{eq:traslinv}
\end{equation}
where $\alpha$ is a constant that we fix to 1 because positive numbers will not be relevant in our discussion. The Fourier Transform of Eq.~\eqref{eq:traslinv} must scale as $(p^2)^{\Delta-d/2}$. Therefore, we only have to fix a constant, which we call $C$. 

To fix $C$, we take the inner product of $|x|^a$ with the Gaussian, in coordinate space on the l.h.s. and in momentum space on the r.h.s., obtaining 
\begin{equation}
    \int_{\mathbb{R}^n} \frac{1}{(2\pi)^{\frac{n}{2}}} e^{-\frac{x^2}{2}} \frac{1}{|x|^a} d^dx = \int_{\mathbb{R}^n} \frac{1}{(2\pi)^{\frac{n}{2}}} e^{-\frac{k^2}{2}} C \frac{1}{|k|^{n-a}} d^dk,
\end{equation}
which can be written in polar coordinates
\begin{equation}
\int_{0}^{\infty} e^{-\frac{r^2}{2}} r^{n-1-a} dr =  \int_{0}^{\infty} e^{-\frac{r^2}{2}} r^{a-1} dr.     
\end{equation}
and solving the integrals on both sides of the equation we have
\begin{equation}
 2^{n-a-2} \Gamma\left(\frac{n - a}{2}\right) = C 2^{a-2} \Gamma\left(\frac{a}{2}\right).
\end{equation}
We neglect the power of 2 since they are positive numbers and hence irrelevant to our discussion. We have obtained that the two-point function in momentum space is
\begin{equation}
    \langle\mathcal{O}(-p)\mathcal{O}(p)\rangle 
    \sim\frac{\Gamma(d/2-\Delta)}{4^{\Delta}\Gamma(\Delta)}(-p^2)^{\Delta-d/2}.
\end{equation}
Consider the tree level amplitude for $J\rightarrow J$ given by
\begin{equation}
    \mathcal{A}=g^2 |J|^2\frac{\Gamma(d/2-\Delta)}{\Gamma(\Delta)}(p^2)^{\Delta-d/2}(-1)^{{\Delta-d/2}},
\end{equation}
and take the forward limit of the cross-section, which is related through the optical theorem to the imaginary part of the forward amplitude $\mathcal{A}_{\text{fwd}}$, given by
\begin{align}
    \text{Im}\mathcal{A}_{\text{fwd}}&=\frac{\Gamma(d/2-\Delta)}{\Gamma(\Delta)}\theta(p_0)\theta(p^2)(p)^{2\Delta-d} \text{Im}\left[(-1)^{{\Delta-d/2}}\right],
\end{align}
that is positive if
\begin{align}
    0<\frac{\Gamma(d/2-\Delta)}{\Gamma(\Delta)}\sin[\pi(d/2-\Delta)]=\frac{\pi}{\Gamma(\Delta)\Gamma(1+\Delta-d/2)},
    \label{eq:impart}
\end{align}
where we used 
\begin{equation}
    \text{Im}\left[(-1)^{{\Delta-d/2}}\right]=\sin[\pi(d/2-\Delta)],\quad \Gamma(1-x)\Gamma(x)\sin(\pi x)=\pi.
\end{equation}
Eq.~\eqref{eq:impart} is positive if
\begin{equation}
    \Delta>\frac{d-2}{2}.
\end{equation}
The case $\Delta=\frac{d-2}{2}$ needs to be treated with care. Using the definition of delta function:
\begin{equation}
    \text{lim}_{a\rightarrow 1}(a-1)\theta(p^2)\frac{1}{(p^2)^{2-a}}=\delta(k^2)
\end{equation}
we get
\begin{equation}
\text{Im}\mathcal{A}_{\text{fwd}}=g^2\pi\theta(p_0)\delta(p^2)>0,
\end{equation}
concluding that 
\begin{equation}
    \Delta\geq \frac{d-2}{2},
    \label{eq:unitarityscalar}
\end{equation}

When it is saturated, i.e. when in a free theory $\Delta$ coincides with the dimension in units of energy of the field $\Delta=(d-2)/2$, we have seen in Eq. (\ref{eq:impart}) that a $\delta(p^2)$ appears. Since $x^2\delta(x^2)=0$, the fact that 
\begin{equation}
    p^2\langle \mathcal{O}(-p)\mathcal{O}(p)\rangle\sim p^2 \delta(p^2)=0
\end{equation}
implies that in coordinate space \cite{Gillioz:2022yze}
\begin{equation}
    \langle \mathcal{O}(x)\partial^2\mathcal{O}(y)\rangle=0~\forall x,y\implies \partial^2\phi(x)=0,
\end{equation}
i.e. the field obeys the equation of motion for a free field, or equivalently, it has the spectral density proportional to a Dirac Delta in the Källén–Lehmann representation of two-point functions.

\subsubsection{Vector operator} For a vector operator in the CFT $\mathcal{O}_\mu$, the external source is $J_\mu$ and the lagrangian is
\begin{equation}
    \mathcal{L}\supset g J_\mu\mathcal{O}^\mu.
\end{equation}

The tensor structure of the two-point function of $\mathcal{O}_\mu$ is fixed by conformal symmetry \cite{Karateev:2020axc}:
\begin{equation}
    \langle O(x)_\mu^\dagger O_\nu(0) \rangle = C  I_{\mu\nu}(x), \quad I_{\mu\nu} \equiv g_{\mu\nu} -2\frac{x_\mu x_\nu}{x^2}.
    \label{eq:twopointvect}
\end{equation}
The two-point function can be evaluated using the differentiation property of the Fourier transform:
\begin{equation}
    FT(xf(x))=-i\partial_k(FT(f(x)).
    \label{eq:diffprop}
\end{equation}
We start with the Fourier transform of the metric tensor term in Eq.~\eqref{eq:twopointvect}, which yields
\begin{equation}
     \sim g_{\mu\nu}\frac{\Gamma(d/2-\Delta)}{4^{\Delta}\Gamma(\Delta)}(k^2)^{\Delta-d/2}\equiv B(k;\Delta,d)g_{\mu\nu},
\end{equation}
while the second term in the tensor structure of Eq.~\eqref{eq:twopointvect} needs more care. First, we note that, since there is an extra power if $x^2$, its Fourier transform can be rewritten using Eq.~\eqref{eq:diffprop} as
\begin{align}
    &-2(i\partial_{k_\mu})(i\partial_{k_\nu})B(k;\Delta+1,d)=2\frac{\Gamma(d/2-\Delta-1)}{4^{\Delta+1}\Gamma(\Delta+1)}\partial_{k_\mu}\partial_{k_\nu}(k^2)^{\Delta+1-d/2}\notag\\&=2\frac{\Gamma(d/2-\Delta-1)}{4^{\Delta+1}\Gamma(\Delta+1)}2(\Delta+1-d/2)\partial_{k_\mu}k_\nu(k^2)^{\Delta-d/2}\notag\\&=4\frac{\Gamma(d/2-\Delta-1)}{4^{\Delta+1}\Gamma(\Delta+1)}(\Delta+1-d/2)\left[g_{\mu\nu}(k^2)^{\Delta-d/2}+2(\Delta-d/2)\frac{k_\nu k_\mu}{k^2}(k^2)^{\Delta-d/2}\right].
    \label{eq:fourtransf}
\end{align}
Putting all together, and using $\Gamma(x+1)=x\Gamma(x)$, we get that the metric tensor term in Eq.~\eqref{eq:fourtransf} has a coefficient given by
\begin{equation}
    (k^2)^{\Delta-d/2}\left(\frac{\Gamma(d/2-\Delta)}{4^{\Delta}\Gamma(\Delta)}-\frac{\Gamma(d/2-\Delta)}{4^{\Delta}\Gamma(\Delta+1)}\right)=(k^2)^{\Delta-d/2}\frac{\Gamma(d/2-\Delta)}{4^{\Delta}\Gamma(\Delta+1)}(\Delta-1),
\end{equation}
while the $k_\mu k_\nu$ term in Eq. (\ref{eq:fourtransf}) has a coefficient given by
\begin{equation}
   -2 (k^2)^{\Delta-d/2}\frac{\Gamma(d/2-\Delta)}{4^{\Delta}\Gamma(\Delta+1)}\Delta-d/2,
\end{equation}
so that the two-point function of Eq. (\ref{eq:twopointvect}) in momentum space is up to positive numbers
\begin{equation}
    (k^2)^{\Delta-d/2}\frac{\Gamma(d/2-\Delta)}{4^{\Delta}\Gamma(\Delta+1)}(\Delta-1)\left(g_{\mu\nu}-2\frac{\Delta-d/2}{\Delta-1}\frac{k_\mu k_\nu}{k^2}\right).
\end{equation}
We evaluate the forward process in the rest frame of the external source, where $\mathbf k=0$:
\begin{align}
    &\text{Im} \mathcal A_{\text{fwd}} \sim -\frac{\Gamma(d/2-\Delta)}{\Gamma(\Delta+1)}(\Delta-1)\left( J \cdot J^\dagger - \frac{2\Delta-d}{\Delta-1}\frac{ |J \cdot k|^2}{k^2} \right) \theta(k^2) \theta(k^0) (k^2)^{\Delta-2}\notag\\&=\frac{\Gamma(d/2-\Delta)}{\Gamma(\Delta+1)}(\Delta-1)\left(|\textbf{J}|^2+\frac{\Delta-d+1}{\Delta-1}|J^0|^2\right),
\end{align}
getting
\begin{equation}
    \Delta\geq d-1,
\end{equation}
that can be generalized for a generic spin tensor to
\begin{equation}
    \Delta\geq d+s-2.
    \label{eq:unitaritytensor}
\end{equation}

Consider the saturated vector unitary bound $\Delta=d-1$. In this case, the two-point function has a vanishing longitudinal component:
\begin{equation}
    p_\mu \langle \mathcal{O}^\mu(-p)\mathcal{O}^\nu(p)\rangle\sim p^\nu-p^\nu=0\implies p_\mu A^\mu|0\rangle=0,
\end{equation}
i.e. $A^\mu$ has only transverse states. In coordinate space this means
\begin{equation}
    \partial_\mu A^\mu(x)|0\rangle=0,
\end{equation}
i.e. a vector with dimension $d-1$ is a conserved current and vice-versa. Note also that any conserved current is a primary operator. In fact it cannot be written as $J_mu=\partial_\mu \phi$, because the current conservation implies $\partial^2\phi=0$, i.e. $\phi$ has a dimension of $(d-2)/2$. $J_\mu$ should then have dimension $d/2$, which lies below the unitary bound.

Thus, when the bound is saturated, the representation of the conformal group is smaller because some descendants are set to zero by equations of motion.

