%------------------------------------------------------------------------
\section*{Brainstorming}
\addcontentsline{toc}{section}{Brainstorming}
%------------------------------------------------------------------------------------------

%----------------------------------------------------------------------------
\subsection*{Flat background: setting \& energy-momentum tensor}
%-------------------------------------------------------------------------

Let the internal space of the fluid be $\mathcal{M}^{(3)}\simeq \R^3$, including also it proper time we have $\mathcal{M}^{(4)}\simeq \R^{1,3}$ isomorphic to Minkowski spacetime.
For the moment, we take flat spacetime with metric $\eta_{\mu\nu}=\text{diag}(-1,1,1,1)$, we will later generalize to nonflat metrics.
The fluid configuration at time $t$ is described by a map
\begin{equation}
    \phi^\mu(t,x^i):  \R^4 \to \mathcal{M}^{(4)}
\end{equation}
%
We define the 3x3 matrix
\begin{equation}
    B^{ij}=\partial_\mu \phi^i \partial^\mu \phi^j \quad i,j=1,2,3,
\end{equation}
The 4-velocity is obtained by demanding the $\phi$ to be indeed comoving coordinates for the fluid and we get
\begin{equation}
    u^\mu=\frac{1}{\sqrt{B}}\epsilon^{\mu\alpha\beta\gamma}\partial_\alpha \phi^1 \partial_\beta \phi^2 \partial_\gamma \phi^3=\frac{1}{6\sqrt{B}}\epsilon^{\mu\alpha\beta\gamma}\epsilon_{ijk}\partial_\alpha \phi^i \partial_\beta \phi^j \partial_\gamma \phi^k,
\end{equation}
satisfying the condition $u^\mu \partial_\mu \phi^i=0$ and the normalization $u_\mu u^\mu=-1$.

Demanding the action be Lorentz invariant and invariant under volume preserving \emph{spatial} diffeomorphisms, we find only two invariants namely
\begin{equation}
    B:=\det B^{ij}, \quad Y:=u^\mu \partial_\mu \phi^0.
\end{equation}
The action is then, for a generic function $F$,
\begin{equation}
    S=\int d^4x F(B,Y).
\end{equation}
%
The energy momentum tensor is obtained by varying the action wrt the cohordinates
\begin{align}
\begin{aligned}
    T^{\mu}{}_{\nu}
    &= \frac{\delta \mathcal{L}}{\delta (\partial_\mu \phi^A)}\,\partial_\nu \phi^A - \mathcal{L}\,\delta^{\mu}{}_{\nu} \\
    &= {\color{red}\bigl(Y \partial_YF - 2B \partial_BF\bigr) u^{\mu} u_{\nu}
       + \bigl(F - 2B \partial_BF\bigr)\,\delta^{\mu}{}_{\nu}}.
\end{aligned}
\end{align}
%
\begin{proof}
Indeed we have
\begin{align}
\begin{aligned}
    \frac{\delta B}{\delta (\partial_\mu \phi^0)} &= 0,
    &\qquad
    \frac{\delta B}{\delta (\partial_\mu \phi^I)} &= B (B^{-1})^{ij} \bigl(\delta^I_i \partial^\mu \phi_j + \delta^I_j \partial^\mu \phi_i\bigr)
    = 2 B (B^{-1})^{Ij} \partial^\mu \phi_j,
    \\[4pt]
    \frac{\delta Y}{\delta (\partial_\mu \phi^0)} &= u^\mu,
    &\qquad
    \frac{\delta Y}{\delta (\partial_\mu \phi^I)} &= \frac{1}{2\sqrt{B}}
    \epsilon^{\lambda\mu\beta\gamma}\epsilon_{Ijk}\partial_\beta\phi^j\partial_\gamma\phi^k\partial_\lambda\phi^0
    - Y \frac{1}{2B}\frac{\delta B}{\delta (\partial_\mu \phi^I)}
    \\
    &&&= \frac{1}{2\sqrt{B}}
    \epsilon^{\lambda\mu\beta\gamma}\epsilon_{Ijk}\partial_\beta\phi^j\partial_\gamma\phi^k\partial_\lambda\phi^0
    - Y (B^{-1})^{Ij} \partial^\mu \phi_j .
\end{aligned}
\end{align}
%
Putting everything together we find
\begin{align}
    T^{\mu}{}_{\nu}
    &= (\partial_YF) u^\mu \partial_\nu \phi^0
    +(\partial_BF)2 B (B^{-1})^{Ij}\,\, \partial^\mu \phi_j \partial_\nu \phi^I
    \\
    &\quad
    + \frac{(\partial_YF)}{2\sqrt{B}}
    \epsilon^{\lambda\mu\beta\gamma}\,\epsilon_{Ijk}\,
    \partial_\beta\phi^j\,\partial_\gamma\phi^k\,\partial_\lambda\phi^0\,\partial_\nu \phi^I
    - (\partial_YF) Y\, (B^{-1})^{Ij} \partial^\mu \phi_j \partial_\nu \phi^I
    - F\,\delta^{\mu}{}_{\nu}.
\end{align}
Raising both indices
\begin{align}
    T^{\mu \nu}
    &= (\partial_YF) u^\mu \partial^\nu \phi^0
    +(\partial_BF)2 B (B^{-1})^{Ij}\,\, \partial^\mu \phi_j \partial^\nu \phi^I
    \\
    &\quad
    + \frac{(\partial_YF)}{2\sqrt{B}}
    \epsilon^{\lambda\mu\beta\gamma}\,\epsilon_{Ijk}\,
    \partial_\beta\phi^j\,\partial_\gamma\phi^k\,\partial_\lambda\phi^0\,\partial^\nu \phi^I
    - (\partial_YF) Y\, (B^{-1})^{Ij} \partial^\mu \phi_j \partial^\nu \phi^I
    - F\,\eta^{\mu\nu}.
\end{align}
Using the identity ($B_{ij}$ is just the projector on the space orthogonal to \(u^\mu\))
\begin{align}
    (B^{-1})^{ij}\,\partial^\mu\phi_j\,\partial_\nu\phi^i &= \delta^{\mu}{}_{\nu} + u^\mu u_\nu
\end{align}
the stress tensor becomes
\begin{align}
    T^{\mu\nu}
    &= (\partial_YF) u^\mu \partial^\nu \phi^0
    + (\partial_YF)\partial_\lambda\phi^0\frac{1}{2\sqrt{B}}
    \epsilon^{\lambda\mu\beta\gamma}\,\epsilon_{Ijk}\,
    \partial_\beta\phi^j\,\partial_\gamma\phi^k\,\partial^\nu \phi^I
    \\
    &+\Big(2B (\partial_BF)- Y(\partial_YF)\Big) (\eta^{\mu\nu} + u^\mu u^\nu)
    - F \eta^{\mu\nu}.
\end{align}
Further decomposing \(\partial_\nu\phi^0 = -Y u_\nu + (\text{part orthogonal to }u_\nu)\), the stress tensor reduces to \questiontag{the first line should vanish, probably symmetrizing}
\begin{align}
    T^{\mu\nu}
    &= (\partial_YF)\left\{ u^\mu \big[\partial^\nu \phi^0 + Y u^\nu\big]
    + \partial_\lambda\phi^0\frac{1}{2\sqrt{B}}
    \epsilon^{\lambda\mu\beta\gamma}\,\epsilon_{Ijk}\,
    \partial_\beta\phi^j\,\partial_\gamma\phi^k\,\partial^\nu \phi^I \right\}
    \\
    &+\Big(2B (\partial_BF)- Y(\partial_YF) -F\Big) \eta^{\mu\nu}
    + u^\mu u^\nu 2\Big(B (\partial_BF)- Y(\partial_YF)\Big).
\end{align}
Matching the expression to that of a perfect fluid
\begin{equation}
    T_{\mu\nu}=(\rho+p)u_\mu u_\nu + p \eta_{\mu\nu},
\end{equation}
and contracting with \(u_\mu u_\nu\) and \(\eta_{\mu\nu}\) we find the energy density and pressure as \questiontag{messed up the minus sign since used flat QFT relation for $T_{\mu\nu}$}
\begin{equation}
    \rho = Y \partial_YF - F, \qquad
    p = F + Y \partial_YF - 2 B \partial_BF.
\end{equation}
We finally get the expresion
\begin{align}
    T^{\mu}{}_{\nu}
    &= 2\bigl(Y (\partial_Y\!F) - B (\partial_B\!F)\bigr) u^{\mu} u_{\nu}
       + \bigl(F + Y (\partial_Y\!F) - 2 B (\partial_B\!F)\bigr)\,\delta^{\mu}{}_{\nu}.
\end{align}
\end{proof}


%----------------------------------------------------------------------------
\subsection*{Curved background: setting \& energy-momentum tensor}
%-------------------------------------------------------------------------
%
Next we include a nontrivial spacetime metric \(g_{\mu\nu}(x)\).
The conditions of the $\phi^A(x^\mu)$ being comoving coordinates for the fluid remain the same, so that the 4-velocity is still identifed by demanding
\begin{equation}
    \frac{d}{d\tau}\phi^I(x^\mu(\tau))=u^\mu \partial_\mu \phi^I=0 \quad I=1,2,3.
\end{equation}
The form of the $4$-velocity is then obtained from the Hodge dual as
\begin{equation}
    u_\mu dx^\mu \propto \star(d\phi^1 \wedge d\phi^2 \wedge d\phi^3).
\end{equation}
Analogously to the flat case we define
\begin{equation}
    B^{ij}=\partial_\mu \phi^i \partial_\nu \phi^j g^{\mu\nu}.
\end{equation}
The invariants are still
\begin{equation}
    B:=\det B^{ij}, \quad Y:=u^\mu \partial_\mu \phi^0.
\end{equation}
The 4-velocity is again explicitly given by
\begin{equation}
    u^\lambda=g^{\lambda\mu}\frac{1}{3!\,\,\sqrt{B}}\sqrt{-g}\,\epsilon_{\mu\alpha\beta\gamma}\,\,\epsilon_{ijk}\,\,\partial^\alpha \phi^i \partial^\beta \phi^j \partial^\gamma \phi^k,
\end{equation}
where $\sqrt{-g}\,\,\epsilon_{\mu\alpha\beta\gamma}$ is the volume form (totally antisymmetric Levi--Civita), with $\epsilon_{0123}=+1$.
%
\begin{proof}
For convenience, we temporarily report the explicit computation.
We have
\begin{align}
    d\phi^1 \wedge d\phi^2 \wedge d\phi^3=\epsilon_{ijk} d\phi^i \otimes d\phi^j \otimes d\phi^k = \epsilon_{ijk}\, \partial_\mu \phi^i \partial_\nu \phi^j \partial_\rho \phi^k \,\,dx^\mu \otimes dx^\nu \otimes dx^\rho.
\end{align}
Raising indices
\begin{align}
    (d\phi^1 \wedge d\phi^2 \wedge d\phi^3)^{\#}
    &= \epsilon_{ijk}\,\, \partial^\mu \phi^i \partial^\nu \phi^j \partial^\rho \phi^k \,\,\partial_\mu \otimes \partial_\nu \otimes \partial_\rho.
\end{align}
The metric volume form on the whole spacetime is
\begin{equation}
    \Omega^g=\sqrt{-g}\,\, dx^0 \wedge dx^1 \wedge dx^2 \wedge dx^3= \sqrt{-g}\,\,\epsilon_{\theta\alpha\beta\gamma}\,\, dx^\theta \otimes dx^\alpha \otimes dx^\beta \otimes dx^\gamma.
\end{equation}
Up to normalization, the velocity dual is then
\begin{align}
    \tilde{u}_\lambda &= \Omega^g\big(\,\partial_\lambda,\, (d\phi^1 \wedge d\phi^2 \wedge d\phi^3)^{\#}\big)
    \\
    &= \sqrt{-g}\,\, \epsilon_{\theta\alpha\beta\gamma} \,\,\epsilon_{ijk}\,\, \partial^\mu \phi^i \partial^\nu \phi^j \partial^\rho \phi^k\,\,
    dx^\theta \otimes dx^\alpha \otimes dx^\beta \otimes dx^\gamma  \big(\partial_\lambda, \, \partial_\mu,\, \partial_\nu,\, \partial_\rho\big)
    \\
    &= \sqrt{-g}\,\, \epsilon_{\lambda\mu\nu\rho} \,\,\epsilon_{ijk}\,\, \partial^\mu \phi^i \partial^\nu \phi^j \partial^\rho \phi^k.
\end{align}
Using the Levi-Civita \emph{tensor} $\sqrt{-g}\,\epsilon_{\alpha\beta\gamma\delta}$ i.e. the volume form, and the tensorial identity (just prove it in normal coordinates)
\begin{align}
    \sqrt{-g}\,\epsilon_{\alpha\beta\gamma\delta}&\,\sqrt{-g}\, \epsilon_{\lambda\mu\nu\rho} \,\,g^{\alpha\lambda} = -3! \,\, g_{\lambda[\alpha} g_{\beta\mu} g_{\nu\rho]}
    \\
    &:= -\Big(
    g_{\beta\mu}\, g_{\gamma\nu}\, g_{\delta\rho}
  + g_{\beta\nu}\, g_{\gamma\rho}\, g_{\delta\mu}
  + g_{\beta\rho}\, g_{\gamma\mu}\, g_{\delta\nu}
  - g_{\beta\mu}\, g_{\gamma\rho}\, g_{\delta\nu}
  - g_{\beta\nu}\, g_{\gamma\mu}\, g_{\delta\rho}
  - g_{\beta\rho}\, g_{\gamma\nu}\, g_{\delta\mu}
\Big),
\end{align}
the (negative) norm is then
\begin{align}
    \tilde{u}_\alpha \,g^{\alpha\lambda}\, \tilde{u}_\lambda
    &= \underbrace{\left(\sqrt{-g}\right)^2 \,g^{\alpha\lambda}\,\,\epsilon_{\alpha\beta\gamma\delta}\,\, \epsilon_{\lambda\mu\nu\rho}}_{=-3! \,\, g_{\lambda[\alpha} g_{\beta\mu} g_{\nu\rho]}}\,\, \underbrace{\epsilon_{ijk}\,\, \epsilon_{pqr}}_{=3!\,\,\delta_p^1\,\delta_q^2\,\delta_r^3\,\,\epsilon_{ijk}} \,\, \partial^\beta \phi^i \partial^\gamma \phi^j \partial^\delta \phi^k \,\,\, \partial^\mu \phi^p \partial^\nu \phi^q \partial^\rho \phi^r 
    \\
    &=
   \, -(3!)^2 \,\, \epsilon_{ijk}\,\,\partial_\mu\phi^1\partial^\mu\phi^i\,\,\partial_\nu\phi^2\partial^\nu\phi^j\,\,\partial_\rho\phi^3\partial^\rho\phi^k  = - (3!)^2 \det B^{ij}.
\end{align}
Raising the index of $\tilde{u}$ and normalizing gives the claimed expression.
\end{proof}

The action is $S=S_{\mathrm{EH}}+S_{\mathrm{fluid}}$ with
\begin{equation}
    S_{\mathrm{fluid}}=\int d^4x\, \sqrt{-g}\, F(B,Y),\quad
    S_{\mathrm{EH}}=\frac{1}{16\pi G}\int d^4x\, \sqrt{-g}\, (R-2\Lambda),
\end{equation}
where $F$ is an arbitrary function, and we included a possible cosmological constant.
The energy-momentum tensor is obtained varying wrt the metric \answertag{Avoid need to symmetrize}
\begin{align}
    T_{\mu\nu}:=-\frac{2}{\sqrt{-g}}\frac{\delta S_{\mathrm{fluid}}}{\delta g^{\mu\nu}}={\color{red} \bigl(3Y (\partial_Y\!F) - 2 B (\partial_B\!F)\bigr)\,u_\mu u_\nu
           + \bigl(F + 2Y (\partial_Y\!F) - 2 B (\partial_B\!F)\bigr)\,g_{\mu\nu}}\,.
\end{align}
That is, matching to the perfect fluid expression, \todotag{Does not match flat case... check!}
\begin{align}
     \rho = Y (\partial_Y\!F) - F, \qquad
    p = F+ 2 Y (\partial_Y\!F) - 2 B (\partial_B\!F).
\end{align}
%
\begin{proof}
We temporarily report the explicit computation, eventually we will cut it out.
Varying the action we have
\begin{align}
    \delta S_{\mathrm{fluid}}
    &= \int d^4x \left(\delta \sqrt{-g}\, F + \sqrt{-g}\, \partial_BF\, \delta B + \sqrt{-g}\, \partial_YF\, \delta Y\right).
\end{align}
We have
\begin{equation}
    \delta \sqrt{-g} = -\frac{1}{2}\, g_{\mu\nu}\,\sqrt{-g}\,\, \delta g^{\mu\nu},
\end{equation}
Using the relation $(B^{-1})^{ij} \partial_\mu\phi^i\,\partial_\nu\phi^j=g_{\mu\nu}+u_\mu u_\nu$ (just projection onto $u_\mu$ orthogonal space) we find
\begin{align}
    \delta B
    &= B (B^{-1})^{ij} \delta B_{ij} = B (B^{-1})^{ij} \partial_\mu\phi^i\,\partial_\nu\phi^j \delta g^{\mu\nu} = B\,\left\{g_{\mu\nu}+u_\mu u_\nu\right\}\,\delta g^{\mu\nu}.
\end{align}
For the variation of \(Y\) we have
{\small
\begin{align}
    \delta Y
    &= \delta\left(u_\lambda g^{\alpha\lambda} \partial_\alpha \phi^0\right)
    \\
     &= \left(u_\lambda \partial_\alpha \phi^0\right)\,\delta g^{\alpha\lambda}-\left(u^\theta \partial_\theta \phi^0\right)\frac{1}{2}B^{-1}\, \delta B
     + \partial_\alpha\phi^0\,g^{\alpha\lambda}\,\frac{1}{3!\,\,\sqrt{B}}\big(\delta\sqrt{-g}\big)\,\epsilon_{\lambda\mu\nu\rho}\,\,\epsilon_{ijk}\,\,\partial_\beta \phi^i \partial_\gamma \phi^j \partial_\delta \phi^k 
     \,g^{\beta\mu}g^{\gamma\nu}g^{\delta\rho}
     \\
     &\quad
     +\partial_\alpha\phi^0\,g^{\alpha\lambda}\,\frac{1}{3!\,\,\sqrt{B}}\sqrt{-g}\,\epsilon_{\lambda\mu\nu\rho}\,\,\epsilon_{ijk}\,\,\partial_\beta \phi^i \partial_\gamma \phi^j \partial_\delta \phi^k 
     \,\,\delta\left(g^{\beta\mu}g^{\gamma\nu}g^{\delta\rho}\right)
     \\
     &= \left\{\partial_\alpha\phi^0\,\, u_\lambda \right\} \delta g^{\alpha\lambda}
     -\frac{Y}{2} \left\{g_{\mu\nu}+u_\mu u_\nu\right\}\,\delta g^{\mu\nu}
     -\frac{Y}{2} g_{\mu\nu}\,\delta g^{\mu\nu}
     \\
     &\quad
     +\partial_\alpha\phi^0\,g^{\alpha\lambda}\,\frac{1}{3!\,\,\sqrt{B}}\sqrt{-g}\,\epsilon_{\lambda\mu\nu\rho}\,\,\epsilon_{ijk}\,\,\partial_\beta \phi^i \partial_\gamma \phi^j \partial_\delta \phi^k 
     \,\,\delta\left(g^{\beta\mu}g^{\gamma\nu}g^{\delta\rho}\right)
\end{align}
}
Splitting $\partial_\alpha\phi^0= -Y u_\alpha + \left[\partial_\alpha\phi^0 + Y u_\alpha\right]$ into parts parallel and orthogonal to $u_\alpha$, we get \todotag{Check 2nd term indeeed cancels}
\begin{align}
    \delta Y
     &=-\frac{Y}{2} \left\{2g_{\mu\nu}+ 3 u_\mu u_\nu\right\}\,\delta g^{\mu\nu}
     \\
     &\quad + \underbrace{\left\{\left[\partial_\alpha\phi^0 + Y u_\alpha\right]\,\, u_\lambda \right\} \delta g^{\alpha\lambda}
     + 
     \partial_\alpha\phi^0\,\,\frac{1}{3!\,\,\sqrt{B}}\sqrt{-g}\,\,\epsilon_{\lambda\mu\nu\rho}\,\,\epsilon_{ijk}\,\,\partial_\beta \phi^i \partial_\gamma \phi^j \partial_\delta \phi^k 
     \,\,g^{\alpha\lambda}\,\,\,\delta\left(g^{\beta\mu}g^{\gamma\nu}g^{\delta\rho}\right)}_{\text{likely to be total derivative or vanish by orthogonality}}.
\end{align}
Putting everything together we find
\begin{align}
    \delta S_{\mathrm{fluid}}
    &= \int d^4x \sqrt{-g} \bigg\{
    -\frac{1}{2} g_{\mu\nu} F
    + (\partial_BF)\, B \left(g_{\mu\nu}+u_\mu u_\nu\right) - (\partial_YF)\, \frac{Y}{2} \left(2g_{\mu\nu}+ 3 \,\,u_\mu u_\nu\right)
    \\
    &\qquad
    +\partial_YF\,\Big(\text{total derivative? hopefully yes}\Big)\bigg\} \delta g^{\mu\nu}.
\end{align}
From which we read off
\begin{align}
    T_{\mu\nu}
    &= -2 \frac{1}{\sqrt{-g}} \frac{\delta S_{\mathrm{fluid}}}{\delta g^{\mu\nu}}
    %
    =g_{\mu\nu} F
    -2 (\partial_BF)\, B \left(g_{\mu\nu}+u_\mu u_\nu\right) + (\partial_YF)\, Y \underbrace{\left(2g_{\mu\nu}+ 3 \,\,u_\mu u_\nu\right)}_{\textcolor{red}{\text{should be g+2uu}}}
    \\
    &=
    \bigl(3Y (\partial_Y\!F) - 2 B (\partial_B\!F)\bigr)\,u_\mu u_\nu
           + \bigl(F + 2Y (\partial_Y\!F) - 2 B (\partial_B\!F)\bigr)\,g_{\mu\nu}.
\end{align}
Matching to the perfect fluid expression $T_{\mu\nu}=(\rho+p)u_\mu u_\nu + p g_{\mu\nu}$ we find again
\begin{equation}
    \rho = Y (\partial_Y\!F) - F, \qquad
    p = F+2Y (\partial_Y\!F) - 2 B (\partial_B\!F).
\end{equation}
\end{proof}


%----------------------------------------------------------------------------
\subsection{Fluctuations: flat case}
%-------------------------------------------------------------------------
Following the notation of \cite{DubovskyGregoireNicolisRattazzi_NullEnergyConditionSuperluminalPropagation_2005}, we consider fluctuations around the background solution
\begin{equation}
    \bar\phi^0 = t, \quad \bar\phi^i = \alpha x^i,
\end{equation}
where \(\alpha\) parametrizes the compression of the fluid.
We then write
\begin{equation}
    \phi^0 = \bar\phi^0 + \pi^0, \quad \phi^i = \bar\phi^i + \pi^i,
\end{equation}
and expand the action up to quadratic order in the fluctuations \(\pi^A\).
%
We have
\begin{equation}
    B^{ij}= \alpha^2 \left(g^{ij} + \partial^i \pi^j + \partial^j \pi^i + \partial_\mu \pi^i \partial^\mu \pi^j\right)
    = \alpha^2 \left(\delta^{ij} + \partial^i \pi^j + \partial^j \pi^i + \partial_\mu \pi^i \partial^\mu \pi^j\right).
\end{equation}
and thus
\begin{align}
    B=\det B^{ij}&= \alpha^6 \det\left(\delta^{ij} + \partial^i \pi^j + \partial^j \pi^i + \partial_\mu \pi^i \partial^\mu \pi^j\right).
\end{align}
Write $X^{ij}=\partial^i \pi^j + \partial^j \pi^i + \partial_\mu \pi^i \partial^\mu \pi^j$, then
\begin{align}
    \det(I+X)=&\exp(\operatorname{tr}\log(I+X))
    \\
    &=\exp(\operatorname{tr}X - \tfrac12 \operatorname{tr}X^{2} + \tfrac13 \operatorname{tr}X^{3} + \ldots)
    \\
    &= 1 + \operatorname{tr}X + \tfrac12(\operatorname{tr}X)^{2} - \tfrac12\operatorname{tr}X^{2} + O(X^3).
\end{align}
Using
\begin{align}
    \operatorname{tr}X &= 2 \partial_i \pi^i + \partial_\mu \pi^i \partial^\mu \pi^i,
    \\
    \operatorname{tr}X^{2} &= \sum_{ij}(\partial^i\pi^j+\partial^j\pi^i)(\partial^i\pi^j+\partial^j\pi^i) + O(\pi^3)
    \\
    &= 2 \partial_i \pi^j \partial_j \pi^i + 2 \partial_i \pi^j \partial_i \pi^j + O(\pi^3),
    \\
    (\operatorname{tr}X)^{2} &= 4 (\partial_i \pi^i)^2 + O(\pi^3),
\end{align}
we finally get
\begin{align}
    B = \alpha^6 \bigg[&1 + 2 \partial_i \pi^i + \partial_\mu \pi^i \partial^\mu \pi^i
    + 2 (\partial_i \pi^i)^2 - \partial_i \pi^j \partial_j \pi^i - \partial_i \pi^j \partial_i \pi^j
    + O(\pi^3)\bigg].
\end{align}
In turn we have
\begin{align}
    B^{-1/2} = \alpha^{-3} \bigg[1 - \partial_i \pi^i - \tfrac12 \partial_\mu \pi^i \partial^\mu \pi^i +\tfrac12 (\partial_i \pi^i)^2 + \tfrac12 \partial_i \pi^j \partial_j \pi^i + \tfrac12 \partial_i \pi^j \partial_i \pi^j 
    + O(\pi^3)\bigg].
\end{align}
%
Next we expand $u^\mu$ up to quadratic order in the fluctuations.
We have\todotag{finish}
\begin{align}
    u^\mu &= \frac{1}{6\sqrt{B}} \epsilon^{\mu\alpha\beta\gamma} \epsilon_{ijk} \partial_\alpha \phi^i \partial_\beta \phi^j \partial_\gamma \phi^k
    \\
    &= \frac{\alpha^3}{6\sqrt{B}} \epsilon^{\mu\alpha\beta\gamma} \epsilon_{ijk} (\delta_\alpha^i + \partial_\alpha \pi^i) (\delta_\beta^j + \partial_\beta \pi^j) (\delta_\gamma^k + \partial_\gamma \pi^k)
    \\
    &= \frac{1}{6} \bigg[1 - \partial_i \pi^i - \tfrac12 \partial_\mu \pi^i \partial^\mu \pi^i +\tfrac12 (\partial_i \pi^i)^2 + \tfrac12 \partial_i \pi^j \partial_j \pi^i + \tfrac12 \partial_i \pi^j \partial_i \pi^j 
    + O(\pi^3)\bigg]
    \\
    &\qquad \epsilon^{\mu\alpha\beta\gamma} \epsilon_{ijk} (\delta_\alpha^i + \partial_\alpha \pi^i) (\delta_\beta^j + \partial_\beta \pi^j) (\delta_\gamma^k + \partial_\gamma \pi^k)
    \\
    &= \underbrace{\frac{1}{6}\epsilon^{\mu\alpha\beta\gamma} \epsilon_{\alpha\beta\gamma}}_{=\delta_0^\mu}+...
\end{align}
Finally the invariant \(Y\) is expanded as \todotag{finish expansion}
\begin{align}
    Y = u^\mu \partial_\mu \phi^0 &= u^\mu (\delta_\mu^0 + \partial_\mu \pi^0) 
    \\
    &= u^0 + u^\mu \partial_\mu \pi^0
\end{align}

The action is then expanded up to quadratic order in the fluctuations as
\begin{align}
    S &= \int d^4x F(B,Y)
    \\
    &= \int d^4x \bigg[ F(\bar{B},\bar{Y}) + \partial_BF(\bar{B},\bar{Y}) \delta B + \partial_YF(\bar{B},\bar{Y}) \delta Y
    \\
    &\qquad + \tfrac12 \partial_B^2F(\bar{B},\bar{Y}) (\delta B)^2 + \partial_B\partial_YF(\bar{B},\bar{Y}) \delta B \delta Y + \tfrac12 \partial_Y^2F(\bar{B},\bar{Y}) (\delta Y)^2
    \bigg].
\end{align}